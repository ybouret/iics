\documentclass[aps]{revtex4}
\usepackage{graphicx}
\usepackage{amssymb,amsfonts,amsmath,amsthm}
\usepackage{chemarr}
\usepackage{bm}
\usepackage{pslatex}
\usepackage{mathptmx}
\usepackage{xfrac}

%% concentration notations
\newcommand{\mymat}[1]{\boldsymbol{#1}}
\newcommand{\mytrn}[1]{{#1}^{\mathsf{T}}}
\newcommand{\myvec}[1]{\overrightarrow{#1}}
\newcommand{\mygrad}{\vec{\nabla}}
\newcommand{\myhess}{\mathcal{H}}


\begin{document}
\title{Differential Bubbles}
\maketitle
Let us assume that we have three points $\vec{M}_-$, $\vec{M}_0$ and $\vec{M}_+$, with
$t_-=\left\vert \myvec{M_0M_-}\right\vert$ and $t_+=\left\vert \myvec{M_0M_+}\right\vert$.
The regular curve approximation is
\begin{equation}
	\vec{M}(t) = \vec{M}_0 + t \partial_t \vec{M} + \frac{1}{2} t^2 \partial_{t}^2 \vec{M}
\end{equation}
so that
\begin{equation}
\left\lbrace
\begin{array}{rcl}
	\myvec{M_0M_-} & = & -t_- \partial_t \vec{M} + \frac{1}{2} t_-^2 \partial_{t}^2 \vec{M}\\
	\myvec{M_0M_-} & = &  t_+ \partial_t \vec{M} + \frac{1}{2} t_+^2 \partial_{t}^2 \vec{M}.\\
\end{array}
\right.
\end{equation}
Setting

\begin{equation}
\left\lbrace
\begin{array}{rcl}
	\vec{V}_- & = & \frac{1}{t_-} \myvec{M_0M_-}\\
	\vec{V}_+ & = & \frac{1}{t_+} \myvec{M_0M_+}\\
\end{array}
\right.
\end{equation}
we shall solve
\begin{equation}
	\left\lbrace
	\begin{array}{rcl}
	\vec{V}_- & = & -\partial_t \vec{M} + \frac{1}{2} t_- \partial_{t}^2 \vec{M}\\
	\vec{V}_+ & = & \partial_t \vec{M} + \frac{1}{2} t_+ \partial_{t}^2 \vec{M}.\\
	\end{array}
\right.
\end{equation}
We find
\begin{equation}
	\left\lbrace
	\begin{array}{rcl}
	\partial_t \vec{M} & = & \frac{1}{t_-+t_+} \left(t_-\vec{V}_+ - t_+ \vec{V}_-\right)\\
	\partial_{t}^2 \vec{M} & = & \frac{2}{t_-+t_+} \left(\vec{V}_- + \vec{V}_+\right) \\
	\end{array}
	\right.
\end{equation}

\end{document}