\documentclass[aps]{revtex4}
\usepackage{graphicx}
\usepackage{amssymb,amsfonts,amsmath,amsthm}
\usepackage{chemarr}
\usepackage{bm}
\usepackage{pslatex}
\usepackage{mathptmx}
\usepackage{xfrac}

%% concentration notations
\newcommand{\mymat}[1]{\boldsymbol{#1}}
\newcommand{\mytrn}[1]{{#1}^{\mathsf{T}}}
\newcommand{\myvec}[1]{\overrightarrow{#1}}
\newcommand{\mygrad}{\vec{\nabla}}
\newcommand{\myhess}{\mathcal{H}}


\begin{document}
\title{Differential Bubbles}
\maketitle

\section{Differential Properties}

\subsection{Tangent Field}
Let us assume that we have three vectors $\vec{M}_-$, $\vec{M}_0$ and $\vec{M}_+$, with
$t_-=\left\vert \myvec{M_0M_-}\right\vert$ and $t_+=\left\vert \myvec{M_0M_+}\right\vert$.
The regular curve approximation is
\begin{equation}
	\vec{M}(t) = \vec{M}_0 + t \partial_t \vec{M} + \frac{1}{2} t^2 \partial_{t}^2 \vec{M}
\end{equation}
so that
\begin{equation}
\left\lbrace
\begin{array}{rcl}
	\myvec{M_0M_-} & = & -t_- \partial_t \vec{M} + \frac{1}{2} t_-^2 \partial_{t}^2 \vec{M}\\
	\myvec{M_0M_-} & = &  t_+ \partial_t \vec{M} + \frac{1}{2} t_+^2 \partial_{t}^2 \vec{M}.\\
\end{array}
\right.
\end{equation}
Setting

\begin{equation}
\left\lbrace
\begin{array}{rcl}
	\vec{V}_- & = & \frac{1}{t_-} \myvec{M_0M_-}\\
	\vec{V}_+ & = & \frac{1}{t_+} \myvec{M_0M_+}\\
\end{array}
\right.
\end{equation}
we shall solve
\begin{equation}
	\left\lbrace
	\begin{array}{rcl}
	\vec{V}_- & = & -\partial_t \vec{M} + \frac{1}{2} t_- \partial_{t}^2 \vec{M}\\
	\vec{V}_+ & = & \partial_t \vec{M} + \frac{1}{2} t_+ \partial_{t}^2 \vec{M}.\\
	\end{array}
\right.
\end{equation}

The first order approximation of the tangent field is
\begin{equation}
	\partial_t \vec{M} =  \frac{1}{t_-+t_+} \left(t_-\vec{V}_+ - t_+ \vec{V}_-\right)
\end{equation}

\subsection{Differential properties}
Let us assume that $\vec{M}(t)$ is a local parametric estimation of a contour.
Then the tangent vector around $\vec{M}_0$ is
\begin{equation}
	\partial_t \vec{M} = 
	 \frac{1}{t_-+t_+} \left\lbrack t_-\left(\frac{1}{t_+}\myvec{M_0M_+}\right) - t_+ \left(\frac{1}{t_-}\myvec{M_0M_-}\right)\right\rbrack
\end{equation}
and the tangent vector is
\begin{equation}
	\vec{\tau} = \frac{1}{\vert \partial_t \vec{M} \vert} \partial_t \vec{M}
\end{equation}
and the normal vector is
\begin{equation}
	\vec{n} = \begin{bmatrix}
		-\tau_y\\
		\tau_x\\
	\end{bmatrix}.
\end{equation}
If $s$ were a curvilinear abscissa then we would have
\begin{equation}
	\partial_s \vec{\tau} = \kappa \vec{n}
\end{equation}
so that
\begin{equation}
	\kappa = \frac{1}{\vert\partial_t \vec{M}\vert} \vec{n}.\partial_t \vec{\tau}
\end{equation}
with
\begin{equation}
	\partial_t \vec{\tau} = \frac{1}{t_-+t_+}
	 \left\lbrack 
	 t_-\left(\frac{1}{t_+}\left(\vec{\tau}_+ - \vec{\tau}_0\right)\right) 
	- 
	t_+ \left(\frac{1}{t_-}\left(\vec{\tau}_- - \vec{\tau}_0\right)\right)
	\right\rbrack.
\end{equation}

So that the differential properties can be determined in two passes:
\begin{enumerate}
	\item Estimate $\partial_t \vec{M}$, deduce $\vec{\tau}$ and $\vec{n}$, and store $\vert\partial_t \vec{M}\vert$,
	\item then estimate $\partial_t \vec{\tau}$ and deduce $\kappa$.
\end{enumerate}

\end{document}