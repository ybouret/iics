\documentclass[11pt]{amsart}
\usepackage{xcolor}
\usepackage{geometry}                % See geometry.pdf to learn the layout options. There are lots.
\geometry{a4paper}                   % ... or a4paper or a5paper or ... 
\usepackage{graphicx}
\usepackage{amssymb}
\usepackage{epstopdf}
\DeclareGraphicsRule{.tif}{png}{.png}{`convert #1 `dirname #1`/`basename #1 .tif`.png}

\title{Technical Details}

\begin{document}
\maketitle

\section{Non uniform differential approximations}

Let $f(x)$ be a function estimated by
\[
	f(x) = f_0 + ax + b x^2
\]
with
\[
	f_\pm = f_0 \pm a \delta_\pm + b \delta_\pm^2
\]
and the conditions
\[
	\delta_\pm > 0 %,\; \delta_- + \delta_+ > 0.
\]
Then
\[
\begin{array}{rcl}
	a & = &
	\dfrac{1}{\delta_-^2\delta_+ + \delta_+^2\delta_-}
	\left[ 
	   \delta_-^2\left(f_+ - f_0 \right)
	 - \delta_+^2\left(f_- - f_0 \right)
	\right] \\
	\\
	& = & \dfrac{1}{\delta_- + \delta_+} 
	\left[
	 \dfrac{\delta_-}{\delta_+} \left(f_+ - f_0 \right)
	-\dfrac{\delta_+}{\delta_-} \left(f_- - f_0 \right)
	\right] \\
	\\
	& = & \dfrac{1}{\delta_- + \delta_+} 
	\left[
	\dfrac{\delta_-}{\delta_+} f_+
	+\left(\dfrac{\delta_+}{\delta_-} - \dfrac{\delta_-}{\delta_+}\right) f_0
	-\dfrac{\delta_+}{\delta_-} f_-
	\right]\\
\end{array}
\]

\section{Differential Arc}
Let $\vec{r}_0$ and $\vec{r}_1$ be two points of a curve
with some respective tangent vector $\vec{\tau}_0$ and $\vec{\tau}_1$, and their
respective curvature $C_0$ and $C_1$.

We assume that $\vec{r}(\mu)$ represent the arc between those two points:
\[
	\vec{r}(\mu) = \vec{r}_0 + \mu \vec{a} + \mu^2 \vec{b} + \mu^3 \vec{c}
\]
We obtain
\[
	\dot{\vec{r}}(\mu) = \vec{a} + 2 \mu \vec{b} + 3 \mu^2 \vec{c}
\]
and
\[
	\ddot{\vec{r}}(\mu) = 2\vec{b} + 6 \mu \vec{c}.
\]
The curvature is
\[
	C(\mu) = \dfrac{\det\left(\dot{\vec{r}},\ddot{\vec{r}}\right)}{\left|\dot{\vec{r}}\right|^3}.
\]

%We need six scalar relations to determine the constant vectors.
%\[
%\begin{array}{rcl}
%	\delta\vec{r} & = &\vec{r}_1 - \vec{r}_0 = \vec{a} + \vec{b} + \vec{c}\\
%	\det(\vec{a},\vec{\tau}_0) & = & 0 \\
%	\det(\vec{a}+2\vec{b}+3\vec{c},\vec{\tau}_1) & = & 0 \\  
%	\dfrac{C_0}{2} & = & \dfrac{\det\left(\vec{a},\vec{b}\right)}{\left|\vec{a}\right|^3}\\
%	\dfrac{C_1}{2} & = & 
%	\dfrac{
%	\det\left(\vec{a},\vec{b}\right)
%	+ 3 \det\left(\vec{a},\vec{c}\right)
%	+ 3 \det\left(\vec{b},\vec{c}\right)
%	}{\left|\vec{a}+2\vec{b}+3\vec{c}\right|^3}
%\end{array}
%\]
%
%With $\vec{U} = (a_x,a_y,b_x,b_y,c_x,c_y)$,
%\[
%\left\lbrack
%	\begin{array}{cccccc}
%	1 & 0 & 1 & 0 & 1 & 0 \\
%	0 & 1 & 0 & 1 & 0 & 1 \\
%	\vec{\tau}_{0,y} & - \vec{\tau}_{0,x} & 0 & 0 & 0 & 0 \\
%	\vec{\tau}_{1,y} & - \vec{\tau}_{1,x} & 2\vec{\tau}_{1,y} & - 2\vec{\tau}_{1,x} & 3\vec{\tau}_{1,y} & -3\vec{\tau}_{1,x} \\
%	\end{array}
%\right\rbrack
%\vec{U} 
%=
%\left[
%\begin{array}{c}
%\delta\vec{r}_x\\
%\delta\vec{r}_y\\
%0\\
%0\\
%\end{array}
%\right]
%\]

\end{document}
%& = &\dfrac{1}{\delta_-^2\delta_+ + \delta_+^2\delta_-}
%	\left[ 
%		\delta_-^2 f_ 
%		+ \left(\delta_+^2 - \delta_-^2\right)f_0 
%		- \delta_+^2 f_-\right] \\