\documentclass[11pt]{article}
\usepackage{amsmath}

\begin{document}
In this project we aim at computing the solution of the following problem
\begin{align}
    0=-\nabla P+\nu \Delta  \mathbf u, \\
    \nabla.{\mathbf u}=0
\end{align}
in the fluid domain.
In the weak formulation, this system becomes:

\begin{equation}
\int {\mathbf v}\nabla P d\Omega+\nu\int \left({\mathbf \nabla} u_x {\mathbf \nabla} v_x + {\mathbf \nabla} u_y {\mathbf \nabla} v_y \right)d\Omega
-\nu \int v_x\partial_n u_x d\Omega_b-\nu \int v_y\partial_n u_y d\Omega_b=0
\end{equation}
with the dynamic boundary conditions $(\sigma.n)n=-P_b+\gamma \kappa$ and $(\sigma.n).\tau=0$, where $n$ and $\tau$ are respectively the vectors normal and tangent to the bubble.
The stress tensor is
\begin{equation}
    \sigma=\left(
    \begin{array}{cc}
    -P +2\nu \partial_x u_x &\nu(\partial_x u_y+\partial_y u_x)\cr
    \nu(\partial_x u_y+\partial_y u_x) &-P +2\nu \partial_y u_y
    \end{array}
    \right)
\end{equation}

We have
\begin{align}
    \partial_n u_x=\partial_x u_x n_x+\partial_y u_x n_y\\
    \partial_n u_y=\partial_x u_y n_x+\partial_y u_y n_y
\end{align}

We then obtain:
\begin{align}
    \partial_x u_x=\left(\partial_n u_x -n_y \partial_y u_x\right)/n_x\\
    \partial_y u_y=\left(\partial_n u_y-n_x \partial_x u_y\right)/n_y
\end{align}
inserting the above relations into the normal and tangential boundary conditions, we get:
\begin{align}
    \partial_n u_x=\frac{1}{2}\left(n_x(P-P_b)+n_y\left(\partial_y u_x-\partial_y u_x\right))\right)\\
    \partial_n u_y=\frac{1}{2}\left(n_y(P-P_b)+n_x\left(\partial_y u_y-\partial_x u_y\right))\right)
\end{align}




\end{document}
