\documentclass[11pt]{amsart}
\usepackage{xcolor}
\usepackage{geometry}                % See geometry.pdf to learn the layout options. There are lots.
\geometry{a4paper}                   % ... or a4paper or a5paper or ... 
\usepackage{graphicx}
\usepackage{amssymb}
\usepackage{epstopdf}
\DeclareGraphicsRule{.tif}{png}{.png}{`convert #1 `dirname #1`/`basename #1 .tif`.png}

\title{Technical Details}

\begin{document}
\maketitle

\section{Non uniform differential approximations}

Let $f(x)$ be a function estimated by
\[
	f(x) = f_0 + ax + b x^2
\]
with
\[
	f_\pm = f_0 \pm a \delta_\pm + b \delta_\pm^2
\]
and the conditions
\[
	\delta_\pm > 0 %,\; \delta_- + \delta_+ > 0.
\]
Then
\[
\begin{array}{rcl}
	a & = &
	\dfrac{1}{\delta_-^2\delta_+ + \delta_+^2\delta_-}
	\left[ 
	   \delta_-^2\left(f_+ - f_0 \right)
	 - \delta_+^2\left(f_- - f_0 \right)
	\right] \\
	\\
	& = & \dfrac{1}{\delta_- + \delta_+} 
	\left[
	 \dfrac{\delta_-}{\delta_+} \left(f_+ - f_0 \right)
	-\dfrac{\delta_+}{\delta_-} \left(f_- - f_0 \right)
	\right] \\
	\\
	& = & \dfrac{1}{\delta_- + \delta_+} 
	\left[
	\dfrac{\delta_-}{\delta_+} f_+
	+\left(\dfrac{\delta_+}{\delta_-} - \dfrac{\delta_-}{\delta_+}\right) f_0
	-\dfrac{\delta_+}{\delta_-} f_-
	\right]\\
\end{array}
\]

\section{Differential Arc}
Let $\vec{r}_0$ and $\vec{r}_1$ be two points of a curve
with some respective tangent vector $\vec{\tau}_0$ and $\vec{\tau}_1$, and their
respective curvature $C_0$ and $C_1$.

We assume that $\vec{r}(\mu)$ represent the arc between those two points:
\[
	\vec{r}(\mu) = \vec{r}_0 + \mu \vec{a} + \mu^2 \vec{b} + \mu^3 \vec{c}
\]
We obtain
\[
	\dot{\vec{r}}(\mu) = \vec{a} + 2 \mu \vec{b} + 3 \mu^2 \vec{c}
\]
and
\[
	\ddot{\vec{r}}(\mu) = 2\vec{b} + 6 \mu \vec{c}.
\]
The curvature is
\[
	C(\mu) = \dfrac{\det\left(\dot{\vec{r}},\ddot{\vec{r}}\right)}{\left|\dot{\vec{r}}\right|^3}
\]
so that
\[
	C(\mu) = \dfrac{2}{\left|\vec{a} + 2 \mu \vec{b} + 3 \mu^2 \vec{c}\right|^3}
	\left[
	\det\left(\vec{a},\vec{b}\right)
	+3\mu\det\left(\vec{a},\vec{c}\right)
	+3\mu^2\det\left(\vec{b},\vec{c}\right)
	\right].
\]
We need a third degree polynomial, otherwise we can only represent
the arcs with the same sign of curvature.
We must solve
\[
	\left\lbrace
	\begin{array}{rcl}
	0 & = & \vec{a}+\vec{b}+\vec{c} - \vec{\delta}\\
	0 & = & 1 - \dfrac{\vec{a}.\vec{\tau}_0}{\left|\vec{a}\right|}\\
	0 & = & 1 - \dfrac{\left(\vec{a}+2\vec{b}+3\vec{c}\right).\vec{\tau}_1}{\left|\vec{a}+2\vec{b}+3\vec{c}\right|}\\
	0 & = & \left|\vec{a}\right|^3\dfrac{C_0}{2} -\det\left(\vec{a},\vec{b}\right)\\
	0 & = & \left|\vec{a}+2\vec{b}+3\vec{c}\right|^3\dfrac{C_1}{2} -
	 \left( 
	 \det\left(\vec{a},\vec{b}\right)
	 + 3 \det\left(\vec{a},\vec{c}\right)
	 + 3 \det\left(\vec{b},\vec{c}\right)
	 \right)\\
	\end{array}
	\right.
\]
Because the gradient of some of those equation vanishes, we can't directly
use the Newton's algorithm.
We shall use a conjugated gradient on the sum of the following objective functions:
\[
	\left\lbrace
	\begin{array}{rcl}
	E_1 & = & \dfrac{1}{2\delta^2}\left(\vec{a}+\vec{b}+\vec{c} - \vec{\delta}\right)^2\\
	E_2 & = & 1 - \dfrac{\vec{a}.\vec{\tau}_0}{\left|\vec{a}\right|}\\
	E_3 & = & 1 - \dfrac{\left(\vec{a}+2\vec{b}+3\vec{c}\right).\vec{\tau}_1}{\left|\vec{a}+2\vec{b}+3\vec{c}\right|}\\
	E_4 & = & \dfrac{\delta^2}{2}\left[ \dfrac{\det\left(\vec{a},\vec{b}\right)}{\left|\vec{a}\right|^3}-\dfrac{C_0}{2}\right]^2\\
	E_5 & = & \dfrac{\delta^2}{2}\left[
	\dfrac
	{\det\left(\vec{a},\vec{b}\right)
	 + 3 \det\left(\vec{a},\vec{c}\right)
	 + 3 \det\left(\vec{b},\vec{c}\right)}
	{\left|\vec{a}+2\vec{b}+3\vec{c}\right|^3}
	 -\dfrac{C_1}{2}\right]^2\\
	\end{array}
	\right.
\]
\end{document}


